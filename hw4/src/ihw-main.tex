\documentclass{amsart}

\usepackage[utf8]{inputenc}
\usepackage[T2A]{fontenc}
\usepackage[english,russian]{babel}
\usepackage{amsthm,amsmath,amsfonts,amssymb}
\usepackage{fullpage}
\usepackage{eufrak}
\usepackage{bbm}

%%% Дополнительная работа с математикой
\usepackage{amsfonts,amssymb,amsthm,mathtools} % AMS
\usepackage{amsmath}
\usepackage{icomma}

%% Шрифты
\usepackage{euscript}	% Шрифт Евклид
\usepackage{mathrsfs}	% Красивый матшрифт

%% Python
\usepackage{pythontex} 


%% Свои команды
\DeclareMathOperator{\lb}{\mathop{lb}}	% логарифм по основанию 2
\DeclareMathOperator{\sgn}{\mathop{sgn}}	% сигнум
\renewcommand{\Im}{\mathop{\mathrm{Im}}\nolimits}	% мнимая часть
\renewcommand{\Re}{\mathop{\mathrm{Re}}\nolimits}	% вещественная часть
\renewcommand{\emptyset}{\varnothing}	% пустое множество
\renewcommand{\le}{\leqslant}	% отечественная версия "меньше или равно"
\renewcommand{\ge}{\geqslant}	% отечественная версия "больше или равно"
\renewcommand{\epsilon}{\varepsilon}	% стандартная "эпсилон"
\renewcommand{\phi}{\varphi}	% стандартная "фи"
\newcommand{\const}{\mathrm{const}}	% константа
\newcommand{\transpose}{^{\mathsf{T}}} % транспонирование

%% Множества чисел
\DeclareMathOperator{\Natural}{\mathbb{N}}	% Натуральные числа
\DeclareMathOperator{\Integer}{\mathbb{Z}}	% Целые числа
\DeclareMathOperator{\Integerp}{\mathbb{Z}_{+}}	% Целые неотрицательные числа
\DeclareMathOperator{\Rational}{\mathbb{Q}}	% Рациональные числа
\DeclareMathOperator{\Real}{\mathbb{R}}	% Вещественные числа
\DeclareMathOperator{\Realp}{\mathbb{R}_{>0}}	% Вещественные положительные числа
\DeclareMathOperator{\Realn}{\mathbb{R}_{<0}}	% Вещественные отрицательные числа
\DeclareMathOperator{\Realnn}{\mathbb{R}_{\ge 0}}	% Вещественные неотрицательные числа
\DeclareMathOperator{\Realnp}{\mathbb{R}_{\le 0}}	% Вещественные неположительные числа
\DeclareMathOperator{\Complex}{\mathbb{C}}	% Комплексные числа

%% Заглавные греческие буквы
\DeclareMathOperator{\Alpha}{\mathrm{A}}	% Альфа
\DeclareMathOperator{\Beta}{\mathrm{B}}	% Вета
\DeclareMathOperator{\Epsilon}{\mathrm{E}}	% Эпсилон
\DeclareMathOperator{\Zeta}{\mathrm{Z}}	% Дзета
\DeclareMathOperator{\Eta}{\mathrm{H}}	% Эта
\DeclareMathOperator{\Iota}{\mathrm{I}}	% Йота
\DeclareMathOperator{\Kappa}{\mathrm{K}}	% Каппа
\DeclareMathOperator{\Mu}{\mathrm{M}}	% Мю
\DeclareMathOperator{\Nu}{\mathrm{N}}	% Ню
\DeclareMathOperator{\Omicron}{\mathrm{O}}	% Омикрон
\DeclareMathOperator{\Rho}{\mathrm{P}}	% Ро
\DeclareMathOperator{\Tau}{\mathrm{T}}	% Тау
\DeclareMathOperator{\Chi}{\mathrm{X}}	% Хи

%% Теория вероятностей
\renewcommand{\Prob}{\mathbb P}	% вероятность
\newcommand{\Expect}{\mathbb E}	% математическое ожидание
\renewcommand{\Variance}{\mathbb D}	% дисперсия
\newcommand{\Entropy}{\mathbb H}	% энтропия
\DeclareMathOperator{\cov}{\mathop{cov}}	% ковариация
\DeclareMathOperator{\supp}{\mathop{supp}}	% носитель
\DeclareMathOperator{\Skewness}{\mathop{Skew}}	% коэффициент асимметрии
\DeclareMathOperator{\Kurtosis}{\mathop{Kurt}}	% коэффициент эксцесса

%%% Статистический анализ
\newcommand*{\moment}[1]{\overline{#1}}	% выборочный момент
\DeclareMathOperator{\hskew}{\mathop{\widehat{Skew}}}	% выборочный коэффициент асимметрии
\DeclareMathOperator{\hkurt}{\mathop{\widehat{Kurt}}}	% выборочный коэффициент эксцесса
%% Однопараметрические распределения
\newcommand*{\chisq}[1]{\chi^2_{#1}}	% Распределение хи-квадрат
\newcommand*{\Stud}[1]{\mathcal{S}_{#1}}	% Распределение Стьюдента
\newcommand*{\Exp}[1]{\mathop{\mathrm{Exp}}(#1)}	% Показательное распределение
\newcommand*{\Bern}[1]{\mathop{\mathrm{Bern}}(#1)}	% Распределение Бернулли
\newcommand*{\Geom}[1]{\mathop{\mathrm{Geom}}(#1)}	% Геометрическое распределение
\newcommand*{\Pois}[1]{\mathop{\mathrm{Pois}}(#1)}	% Распределение Пуассона
%% Двухпараметрические распределения
\newcommand*{\FS}[2]{\mathcal{F}_{#1, #2}}	% Распределение Фишера-Снедекора
\newcommand*{\Norm}[2]{\mathcal{N}(#1, #2)}	% Нормальное распределение
\newcommand*{\Unif}[2]{\mathcal{U}(#1, #2)}	% Равномерное распределение
\newcommand*{\DE}[2]{\mathop{\mathrm{DE}}(#1, #2)}	% Распределение Лапласа
\newcommand*{\Cauchy}[2]{\mathop{\mathrm{C}}(#1, #2)}	% Распределение Коши
\newcommand*{\Binom}[2]{\mathop{\mathrm{Binom}}(#1, #2)}	% Биномиальное распределение
\newcommand*{\Betadist}[2]{\mathop{\mathrm{Beta}}(#1, #2)}	% Бета-распределение
\newcommand*{\Gammadist}[2]{\mathop{\mathrm{Gamma}}(#1, #2)}	% Гамма-распределение
%% Ажурные и готические буквы
\newcommand*{\Acl}{\mathcal{A}}	% A красивое
\newcommand*{\Ccl}{\mathcal{C}}	% C красивое
\newcommand*{\Fcl}{\mathcal{F}}	% F красивое
\newcommand*{\Icl}{\mathcal{I}}	% I красивое
\newcommand*{\Kcl}{\mathcal{K}}	% K красивое
\newcommand*{\Pcl}{\mathcal{P}}	% P красивое
\newcommand*{\Ycl}{\mathcal{Y}}	% Y красивое
\newcommand*{\Afr}{\mathfrak{A}}	% A готическое
\newcommand*{\Bfr}{\mathfrak{B}}	% B готическое
\newcommand*{\Ffr}{\mathfrak{F}}	% F готическое
\newcommand*{\Kfr}{\mathfrak{K}}	% K готическое
\newcommand*{\Xfr}{\mathfrak{X}}	% X готическое
%% Теория оценивания
\newcommand*{\ind}[1]{\mathbbm{1}_{\lbrace #1 \rbrace}}	% индикаторная функция
\newcommand*{\bias}[2]{\mathop{\mathrm{bias}}\nolimits_{#1}(#2)}	% смещение

%% Перенос знаков в формулах (по Львовскому)
\newcommand*{\hm}[1]{#1\nobreak\discretionary{}
	{\hbox{$\mathsurround=0pt #1$}}{}}

%%% Работа с картинками
\usepackage{graphicx,xcolor}	% Для вставки рисунков
\graphicspath{{images/}{images2/}}	% папки с картинками
\setlength\fboxsep{3pt}	% Отступ рамки \fbox{} от рисунка
\setlength\fboxrule{1pt}	% Толщина линий рамки \fbox{}
\usepackage{wrapfig}	% Обтекание рисунков и таблиц текстом
\RequirePackage{caption}
\DeclareCaptionLabelSeparator{defffis}{ --- }
\captionsetup{justification=centering,labelsep=defffis}
\usepackage{float}
\usepackage{tikz}
\usepackage{pgfplots}
\pgfplotsset{compat=newest}
\usetikzlibrary{patterns}
\usetikzlibrary{calc}
\usepgfplotslibrary{fillbetween}
\usepackage{svg}


%%% Работа с таблицами
\usepackage{array,tabularx,tabulary,booktabs}	% Дополнительная работа с таблицами
\usepackage{longtable}	% Длинные таблицы
\usepackage{multirow}	% Слияние строк в таблице
\usepackage{makecell}
\usepackage{multicol}
\usepackage{diagbox}


\renewcommand{\qedsymbol}{}

\newtheorem{problem}{Задание}

\begin{document}
\newcommand{\problemset}[1]{
	\begin{center}
		\Large #1
	\end{center}
}

\begin{tabbing}
	\hspace{11cm} \= Студент: \= Рыжиков Иван \\	% не забудьте исправить, студент Вы или студентка :)
	% (а то некоторые забывают)
	\> Группа: \> 2381 \\	% Здесь меняете № группы
	\> Вариант: \> 18 \\		% А здесь меняете № варианта
	\> Дата: \> \today		% А вот здесь ничего не меняем!!!
\end{tabbing}
\hrule
\vspace{1cm}	% в данном файле меняем только Пол, Фамилию Имя, № группы и № варианта
\problemset{Теория вероятностей и математическая статистика}
\problemset{Индивидуальное домашнее задание №3}	% поменяйте номер ИДЗ

\renewcommand*{\proofname}{Решение}


Случайный вектор ($\xi$, $\eta$) имеет равномерное распределение в области $D$:
\[
    D=\begin{pmatrix}4x-2y\geqslant2,\\x\leqslant3,y\geqslant1\end{pmatrix}
\]
$\zeta=2\xi^4-2$, $\nu=\lfloor5\eta\rfloor$, $\mu= -8\xi+4\eta$.

\begin{sympycode}
x, y, z = symbols('x y z')
xi, eta = symbols('xi eta')
x2 = 3
y1 = 1
line_eq = 4*x - 2*y - 2
yto = solve(line_eq, y)[0]
xfrom = solve(line_eq, x)[0]
x1 = xfrom.subs(y, y1)
y2 = yto.subs(x, x2)
D = And(line_eq >= 0, x <= x2, y >= y1)
x_interval = Interval(x1, x2)
y_interval = Interval(y1, y2)
x_interval_by_y = Interval(xfrom, x2)
y_interval_by_x = Interval(y1, yto)
zeta = 2 * xi ** 4 - 2
nu = floor(5 * eta)
mu = -8 * xi + 4 * eta
\end{sympycode}

\begin{figure}[h!]
    \centering
    \begin{tikzpicture}
        \begin{axis}[
                xlabel=$x$,
                ylabel=$y$,
                xmin=-1.5, xmax=4.5,
                ymin=-1.5, ymax=8.5,
                axis lines=middle,
                axis line style={->},
                % ticks=none,
                clip=true,
                xtick={-1,0,1,2,3,4},
                ytick={-1,0,1,2,3,4,5,6,7,8},
                axis equal,
                grid=both,
            ]
            \addplot[fill=blue!10] coordinates {(1,1) (3, 5) (3, 1) (1,1)};
            % \addplot[domain=1:3, blue, dashed] {2*x-1};
            \node[blue, right, rotate=63.435] at (axis cs: 1, 1.7) {$2x-1y-1=0$};
        \end{axis}
    \end{tikzpicture}
    \caption{Область $D$}
    \label{fig:D}
\end{figure}





%%%%%%%%%%%%%% ЗАДАНИЕ №1 %%%%%%%%%%%%%%
%% Условие задания №1
\begin{problem}
Haйти $p_{\xi, \eta} ( x, y) $, функции и плотности распределения компонент. Построить графики функций распределений $F_\xi(x)$ и $F_\eta(y)$. Будут ли компоненты независимыми?
\end{problem}

\begin{sympycode}
C = symbols("C")
p_xi_eta = Piecewise((C, True), (0, ~D))
int1 = integrate(integrate(p_xi_eta, (y, y_interval_by_x)), (x, x_interval))
C1 = solve(int1 - 1, C)[0]
\end{sympycode}

%% Решение задания №1
\begin{proof}
    Равномерное распределение задаётся следующей плотностью:
    \[
        p_{\xi, \eta} (x, y) = \begin{cases}
            C, & \text{если } (x, y) \in D, \\
            0, & \text{иначе}.
        \end{cases}
    \]
    Найдём константу $C$:
    \[
        \begin{aligned}
            1 & = \iint\limits_{D} p_{\xi, \eta} (x, y) \, dx \, dy
            = \int\limits_{\sympy{x_interval.start}}^{\sympy{x_interval.end}} \, dx \int\limits_{\sympy{y_interval_by_x.start}}^{\sympy{y_interval_by_x.end}} C \, dy
            = \sympy{int1},                                         \\
            C & = \sympy{C1}.
        \end{aligned}
    \]

    Таким образом, плотность распределения равна:

    \[
        \begin{aligned}
            p_{\xi, \eta} (x, y)
             & =
            \begin{cases}
                \sympy{C1}, & \text{если } x \in \sympy{x_interval}, y \in \sympy{y_interval_by_x} \\
                0,          & \text{иначе}.
            \end{cases}
            \\
             & =
            \begin{cases}
                \sympy{C1}, & \text{если } y \in \sympy{y_interval}, x \in \sympy{x_interval_by_y} \\
                0,          & \text{иначе}.
            \end{cases}
        \end{aligned}
    \]
\begin{sympycode}
p_xi_eta = C1
p_xi = simplify(integrate(p_xi_eta, (y, y_interval_by_x.start, y_interval_by_x.end)))
p_eta = simplify(integrate(p_xi_eta, (x, x_interval_by_y.start, x_interval_by_y.end)))
\end{sympycode}

    Найдём плотность распределения компонент:

    \[
        \begin{aligned}
            p_{\xi} (x)  &
            = \int\limits_{-\infty}^{+\infty} p_{\xi, \eta} (x, y) \, dy
            = \int\limits_{\sympy{y_interval.start}}^{\sympy{y_interval.end}} \sympy{p_xi_eta} \, dy
            = \sympy{p_xi}, \\
            p_{\eta} (y) &
            = \int\limits_{-\infty}^{+\infty} p_{\xi, \eta} (x, y) \, dx
            = \int\limits_{\sympy{x_interval.start}}^{\sympy{x_interval.end}} \sympy{p_xi_eta} \, dx
            = \sympy{p_eta}.
        \end{aligned}
    \]

    Итого,

    \[
        \begin{aligned}
            p_{\xi} (x)  &
            = \begin{cases}
                  \sympy{p_xi}, & x \in \sympy{x_interval}, \\
                  0,            & \text{иначе},
              \end{cases} \\
            p_{\eta} (y) &
            = \begin{cases}
                  \sympy{p_eta}, & y \in \sympy{y_interval}, \\
                  0,             & \text{иначе}.
              \end{cases}
        \end{aligned}
    \]

    \begin{sympycode}
F_xi = integrate(p_xi, (x, x_interval.start, x))
F_eta = integrate(p_eta, (y, y_interval.start, y))
  \end{sympycode}
    Найдём функции распределения компонент:

    \[
        \begin{aligned}
            F_\xi
            = \int\limits_{-\infty}^{x} p_{\xi} (t) \, dt
             & = \begin{cases}
                     0,            & x < \sympy{x_interval.start}, \\
                     \sympy{F_xi}, & x \in \sympy{x_interval},     \\
                     1,            & x > \sympy{x_interval.end},
                 \end{cases}  \\
            F_\eta
            = \int\limits_{-\infty}^{y} p_{\eta} (t) \, dt
             & = \begin{cases}
                     0,             & y < \sympy{y_interval.start}, \\
                     \sympy{F_eta}, & y \in \sympy{y_interval},     \\
                     1,             & y > \sympy{y_interval.end}.
                 \end{cases}
        \end{aligned}
    \]
    \begin{figure}[h!]
        \centering
        \begin{tikzpicture}
            \begin{axis}[
                    xlabel=$x$,
                    ylabel=$F_\xi(x)$,
                    xmin=-1.5, xmax=4.5,
                    ymin=-0.5, ymax=1.5,
                    axis lines=middle,
                    axis line style={->},
                    % ticks=none,
                    clip=true,
                    xtick={-1,0,1,2,3,4},
                    ytick={0,1},
                    grid=both,
                ]
                \addplot[domain=-1.5:1, blue, solid] {0};
                \addplot[domain=1:3, blue, solid] {x^2 / 4 - x/2 + 1/4};
                \addplot[domain=3:4.5, blue, solid] {1};
            \end{axis}
        \end{tikzpicture}
        \caption{График функции распределения $F_\xi(x)$}
        \label{fig:F_xi}
    \end{figure}

    \begin{figure}[h!]
        \centering
        \begin{tikzpicture}
            \begin{axis}[
                    xlabel=$y$,
                    ylabel=$F_\eta(y)$,
                    xmin=-1.5, xmax=8.5,
                    ymin=-0.5, ymax=1.5,
                    axis lines=middle,
                    axis line style={->},
                    % ticks=none,
                    clip=true,
                    xtick={-1,0,1,2,3,4,5,6,7,8},
                    ytick={0,1},
                    grid=both,
                ]
                \addplot[domain=-1.5:1, red, solid] {0};
                \addplot[domain=1:5, red, solid] {-x ^ 2 / 16 + 5 * x / 8 - 9 / 16};
                \addplot[domain=5:8.5, red, solid] {1};
            \end{axis}
        \end{tikzpicture}
        \caption{График функции распределения $F_\eta(y)$}
        \label{fig:F_eta}
    \end{figure}

    \begin{sympycode}
p_xi_eta_prod = p_xi * p_eta
  \end{sympycode}
    Проверка компонент на независимость:

    Для точки $(x, y) \in D$ плотность распределения компонент равна
    \[
        \begin{aligned}
            p_{\xi, \eta} (x, y) & = \sympy{p_xi_eta},                                                                          \\
            p_{\xi, \eta} (x, y) & = p_{\xi} (x) \cdot p_{\eta} (y) = \sympy{p_xi_eta_prod} = \sympy{p_xi_eta_prod.simplify()}.
        \end{aligned}
    \]

    Таким образом, компоненты не являются независимыми.

\end{proof}


\newpage


\begin{problem}
Haйти распределения случайных величин $\zeta$ и $\nu.$ Вычислить $\Expect\zeta$, $\Variance\zeta$, $\Expect\nu$ и $\Variance\nu$. Построить графики функций распределений $F_\zeta(z)$ и $F_\nu(n).$
\end{problem}
\begin{proof}

    $\zeta=\sympy{zeta}$, $\nu=\sympy{nu}$.

Найдём распределение случайной величины $\zeta$:

\begin{sympycode}
zeta1=zeta.subs(xi, x)
supp_zeta = ImageSet(Lambda(x, zeta1), x_interval)
supp_zeta1 = supp_zeta.simplify()
\end{sympycode}

Носитель случайной величины $\zeta$:
\[
    \supp \zeta = \sympy{supp_zeta} = \sympy{supp_zeta1}
\]

\begin{figure}[h!]
    \centering
    \begin{tikzpicture}
        \begin{axis}[
                xlabel=$\xi$,
                ylabel=$\zeta$,
                xmin=-0.5, xmax=3.5,
                ymin=0, ymax=170,
                axis lines=middle,
                axis line style={->},
                % ticks=none,
                % clip=true,
                % xtick={-1,-0.5,0,0.5,1},
                % ytick={-3,-2,-1,0,1,2,3},
                grid=both,
            ]
            \addplot[domain=1:3, blue, solid] {2 * x^4 - 2};
        \end{axis}
    \end{tikzpicture}
    \caption{График величины $\zeta$ в зависимости от $\xi$}
    \label{fig:zeta}
\end{figure}
\begin{sympycode}
x_by_z = solve(z - zeta1, x)[-1]
\end{sympycode}
1) $\zeta \in \sympy{supp_zeta1} \Rightarrow \xi \in \sympy{x_interval}$
\[
    A = \sympy{x_interval} \Rightarrow
    \begin{cases}
        g(x) = \sympy{zeta1}, \\
        g^{-1}(z) = \sympy{x_by_z}.
    \end{cases}
\]
\begin{sympycode}
x_by_z_derive = diff(x_by_z, z)
p_xi_sub_xz = p_xi.subs(x, x_by_z)
p_zeta = (x_by_z_derive * p_xi_sub_xz).simplify()
\end{sympycode}
\[
    p_\zeta
    = |(g^{-1}(z))'| \cdot p_{\xi} (g^{-1}(z))
    = \sympy{x_by_z_derive} \cdot \left(\sympy{p_xi_sub_xz}\right)
    = \sympy{p_zeta},
    z \in \sympy{supp_zeta1}.
\]

Итого, плотность распределения случайной величины $\zeta$:

\[
    p_\zeta (z)
    = \begin{cases}
        \sympy{p_zeta}, & z \in \sympy{supp_zeta1}, \\
        0,              & \text{иначе}.
    \end{cases}
\]
Мат. ожидание и дисперсия случайной величины $\zeta$:
\begin{sympycode}
Expect_zeta = integrate(z * p_zeta, (z, supp_zeta1.start, supp_zeta1.end))
Expect_zeta1 = Expect_zeta.simplify()
Expect_zeta_square = integrate((z) ** 2 * p_zeta, (z, supp_zeta1.start, supp_zeta1.end))
Variance_zeta1 = Expect_zeta_square - Expect_zeta ** 2
\end{sympycode}
\[
    \begin{aligned}
        \Expect\zeta
         & = \int\limits_{\sympy{supp_zeta1.start}}^{\sympy{supp_zeta1.end}} z \cdot p_\zeta (z) \, dz
        = \int\limits_{\sympy{supp_zeta1.start}}^{\sympy{supp_zeta1.end}} z \cdot \sympy{p_zeta} \, dz
        = \sympy{Expect_zeta1},                                                                        \\
        \Variance\zeta
         & = \Expect\zeta^2 - \left(\Expect\zeta\right)^2
        = \int\limits_{\sympy{supp_zeta1.start}}^{\sympy{supp_zeta1.end}} z^2 \cdot p_\zeta (z) \, dz - \left(\Expect\zeta\right)^2
        = \sympy{Expect_zeta_square} - \left(\sympy{Expect_zeta}\right)^2
        = \sympy{Variance_zeta1}.
    \end{aligned}
\]

По плотности вычислим функцию распределения \(\zeta\):

\begin{sympycode}
p_zeta_piece = Piecewise((p_zeta, supp_zeta1.as_relational(z)), (0, True))
F_zeta2 = integrate(p_zeta, (z, supp_zeta1.start, z)).simplify()
F_zeta1 = Piecewise((0, z <= supp_zeta1.start),
                    (F_zeta2, And(supp_zeta1.start < z, z <= supp_zeta1.end)), 
                    (1, True))
F_zeta1 = piecewise_exclusive(F_zeta1)
\end{sympycode}

\[
    F_\zeta (z) =
    \int\limits_{\sympy{supp_zeta1.start}}^{z} p_\zeta (z) \, dz =
    \int\limits_{\sympy{supp_zeta1.start}}^{z} \sympy{p_zeta} \, dz =
    \sympy{F_zeta2}
    z \in \sympy{supp_zeta1}.
\]

\[F_\zeta(z) = \sympy{F_zeta1}\]

\begin{figure}[h!]
    \centering
    \begin{tikzpicture}
        \begin{axis}[
                xlabel=$z$,
                ylabel=$F_\zeta(z)$,
                % xmin=-0.5, xmax=170,
                ymin=-0.4, ymax=1.4,
                axis lines=middle,
                axis line style={->},
                % ticks=none,
                % clip=true,
                % xtick={-1,-0.5,0,0.5,1},
                % ytick={-3,-2,-1,0,1,2,3},
                grid=both,
            ]
            \addplot[domain=-40:0, blue, solid] {0};
            \addplot[domain=0:160, blue, solid]
            {(2^(1/2) * x + 2 * (x+2)^(1/2)*(-2^(3/4) * (x+2)^(1/4) + 1 ) + 2 * (2)^(1/2))/(8* (x+ 2)^(1/2))};
            \addplot[domain=160:200, blue, solid] {1};
        \end{axis}
    \end{tikzpicture}
    \caption{График функции распределения $F_\zeta(z)$}
    \label{fig:F_zeta}
\end{figure}


    
Найдём распределение случайной величины $\nu$:

\begin{sympycode}
n = symbols("n")
nu1 = nu.subs(eta, y)
supp_nu = ImageSet(Lambda(y, nu1), y_interval)
supp_nu1 = FiniteSet(1,2,3,4,5)
\end{sympycode}

Носитель случайной величины $\nu$:
\[
    \supp \nu = \sympy{supp_nu} = \sympy{supp_nu1}
\]

\begin{figure}[h!]
    \centering
    \begin{tikzpicture}
        \begin{axis}[
                xlabel=$\eta$,
                ylabel=$\nu$,
                xmin=-0.5, xmax=6.5,
                ymin=-0.5, ymax=6.5,
                axis lines=middle,
                axis line style={->},
                yscale=1,
                xscale=1,
                % ticks=none,
                % clip=true,
                xtick={0, 1, 2, 3, 4, 5, 6},
                ytick={0, 1, 2, 3, 4, 5, 6},
                grid=both,
                axis equal,
            ]
            \addplot[domain=1:5, blue, solid] (1,1) -- (2,1);
            \addplot[mark=*, mark size=2pt, blue] coordinates {(1,1)};
            \addplot[domain=1:5, blue, solid] (2,2) -- (3,2);
            \addplot[mark=*, mark size=2pt, blue] coordinates {(2,2)};
            \addplot[domain=1:5, blue, solid] (3,3) -- (4,3);
            \addplot[mark=*, mark size=2pt, blue] coordinates {(3,3)};
            \addplot[domain=1:5, blue, solid] (4,4) -- (5,4);
            \addplot[mark=*, mark size=2pt, blue] coordinates {(4,4)};

            \addplot[mark=*, mark size=2pt, blue] coordinates {(5,5)};

        \end{axis}
    \end{tikzpicture}
    \caption{График величины $\nu$ в зависимости от $\eta$}
    \label{fig:nu}
\end{figure}

\begin{sympycode}
y_by_t1 = Interval(1,2)
y_by_t2 = Interval(2,3)
y_by_t3 = Interval(3,4)
y_by_t4 = Interval(4,5)
y_by_t5 = 5
p_nu1 = integrate(p_eta, (y, y_by_t1.start, y_by_t1.end))
p_nu2 = integrate(p_eta, (y, y_by_t2.start, y_by_t2.end))
p_nu3 = integrate(p_eta, (y, y_by_t3.start, y_by_t3.end))
p_nu4 = integrate(p_eta, (y, y_by_t4.start, y_by_t4.end))
p_nu5 = integrate(p_eta, (y, y_by_t5, y_by_t5))
\end{sympycode}

Величина $\nu$ дискретная. Поэтому для каждого значения $\nu$ найдём вероятность. Для этого разобьём носитель $\eta$ на отрезки.

1) $\nu = 1 \Rightarrow \eta \in \sympy{y_by_t1}$
\[
    A_1 = \sympy{y_by_t1} \Rightarrow
    {p_\nu}_1
    = \int\limits_{\sympy{y_by_t1.start}}^{\sympy{y_by_t1.end}} p_{\eta} (y) \, dy
    = \int\limits_{\sympy{y_by_t1.start}}^{\sympy{y_by_t1.end}} \sympy{p_eta} \, dy
    = \sympy{p_nu1}.
\]

2) $\nu = 2 \Rightarrow \eta \in \sympy{y_by_t2}$
\[
    A_2 = \sympy{y_by_t2} \Rightarrow
    {p_\nu}_2
    = \int\limits_{\sympy{y_by_t2.start}}^{\sympy{y_by_t2.end}} p_{\eta} (y) \, dy
    = \int\limits_{\sympy{y_by_t2.start}}^{\sympy{y_by_t2.end}} \sympy{p_eta} \, dy
    = \sympy{p_nu2}.
\]

3) $\nu = 3 \Rightarrow \eta \in \sympy{y_by_t3}$
\[
    A_3 = \sympy{y_by_t3} \Rightarrow
    {p_\nu}_3
    = \int\limits_{\sympy{y_by_t3.start}}^{\sympy{y_by_t3.end}} p_{\eta} (y) \, dy
    = \int\limits_{\sympy{y_by_t3.start}}^{\sympy{y_by_t3.end}} \sympy{p_eta} \, dy
    = \sympy{p_nu3}.
\]
4) $\nu = 4 \Rightarrow \eta \in \sympy{y_by_t4}$
\[
    A_4 = \sympy{y_by_t4} \Rightarrow
    {p_\nu}_4
    = \int\limits_{\sympy{y_by_t4.start}}^{\sympy{y_by_t4.end}} p_{\eta} (y) \, dy
    = \int\limits_{\sympy{y_by_t4.start}}^{\sympy{y_by_t4.end}} \sympy{p_eta} \, dy
    = \sympy{p_nu4}.
\]
5) $\nu = 5 \Rightarrow \eta \in \sympy{y_by_t5}$
\[
    A_5 = \sympy{y_by_t5} \Rightarrow
    {p_\nu}_5
    = \int\limits_{\sympy{y_by_t5}}^{\sympy{y_by_t5}} p_{\eta} (y) \, dy
    = \int\limits_{\sympy{y_by_t5}}^{\sympy{y_by_t5}} \sympy{p_eta} \, dy
    = \sympy{p_nu5}.
\]

Итого, вероятность получить каждое из значений случайной величины $\nu$:
\begin{sympycode}
p_nu = Piecewise((p_nu1, Eq(n, 1)), (p_nu2, Eq(n, 2)), (p_nu3, Eq(n, 3)), (p_nu4, Eq(n, 4)), (p_nu5, Eq(n, 5)), (0, True))
\end{sympycode}
\begin{table}[h!]
    \centering
    \caption{Распределение случайной величины $\nu$}
    \begin{tabular}{|c|c|c|c|c|c|}
        \hline
        $\nu$   & 1               & 2               & 3               & 4               & $\sum$ \\
        \hline
        $\Prob$ & $\sympy{p_nu1}$ & $\sympy{p_nu2}$ & $\sympy{p_nu3}$ & $\sympy{p_nu4}$ & 1      \\
        \hline
    \end{tabular}
\end{table}

Найдём мат. ожидание и дисперсию случайной величины $\nu$:

\begin{sympycode}
Expect_nu = Sum(n * p_nu, (n, 1, 4))
Expect_nu1 = Expect_nu.doit()
Expect_nu_square = Sum(n ** 2 * p_nu, (n, 1, 4)).doit()
Variance_nu1 = Expect_nu_square - Expect_nu1 ** 2
\end{sympycode}
\[
    \begin{aligned}
        \Expect\nu
         & = \sum\limits_{n=1}^{4} n \cdot p_\nu (n)
        = \sympy{Expect_nu1},                         \\
        \Variance\nu
         & = \Expect\nu^2 - \left(\Expect\nu\right)^2
        = \sum\limits_{n=1}^{4} n^2 \cdot p_\nu (n) - \left(\Expect\nu\right)^2
        = \sympy{Expect_nu_square} - \left(\sympy{Expect_nu1}\right)^2
        = \sympy{Variance_nu1}.
    \end{aligned}
\]

Построим графики функций распределения $F_\nu(n)$:

\begin{sympycode}
F_nu1 = Piecewise(
    (0, n <= 1),
    (Sum(p_nu, (n, 1, 1)).doit(), n <= 2),
    (Sum(p_nu, (n, 1, 2)).doit(), n <= 3),
    (Sum(p_nu, (n, 1, 3)).doit(), n <= 4),
    (Sum(p_nu, (n, 1, 4)).doit(), n <= 5),
    (1, True)
)
F_nu1 = piecewise_exclusive(piecewise_fold(F_nu1))
\end{sympycode}

\[
    F_\nu (n) = \sum\limits_{x=1}^{n} p_\nu (x)
\]

\[
    F_\nu (n) = \sympy{F_nu1}
\]

\begin{figure}
    \centering
    \begin{tikzpicture}
        \begin{axis}[
                xlabel=$n$,
                ylabel=$F_\nu(n)$,
                xmin=-0.5, xmax=6.5,
                ymin=-0.4, ymax=1.4,
                axis lines=middle,
                axis line style={->},
                % yscale=1,
                % xscale=1,
                % ticks=none,
                % clip=true,
                xtick={0, 1, 2, 3, 4, 5, 6},
                ytick={0, 1},
                grid=both,
            ]
            \addplot[domain=1:5, blue, solid] (-1,0) -- (1,0);
            \addplot[domain=1:5, blue, solid] (1,7/16) -- (2,7/16);
            \addplot[domain=1:5, blue, solid] (2,0.75) -- (3,0.75);
            \addplot[domain=1:5, blue, solid] (3,15/16) -- (4,15/16);
            \addplot[domain=1:5, blue, solid] (4,1) -- (6,1);

        \end{axis}
    \end{tikzpicture}
    \caption{График функции распределения $F_\nu(n)$}
    \label{fig:F_nu}
\end{figure}

\end{proof}

\newpage

\begin{problem}
Вычислить вектор математических ожиданий, поcтроить ковариационную и корреляционную матрицы для вектора $(\xi,\eta)$. Найти условное распределение $\xi$ при условии $\eta$. Вычислить $\Expect(\xi|\eta)$ и $\Variance(\xi|\eta)$.
\end{problem}
\begin{proof}
    Найдем мат. ожидание случайной величины $(\xi,\eta)$:
    \begin{sympycode}
# Expect_xi = integrate(x * p_xi_eta, (x, x_interval_by_y.start, x_interval_by_y.end), (y, y_interval.start, y_interval.end))
# Expect_eta = integrate(y * p_xi_eta,  (y, y_interval_by_x.start, y_interval_by_x.end), (x, x_interval.start, x_interval.end)).simplify()
Expect_xi = integrate(x * p_xi, (x, x_interval.start, x_interval.end))
Expect_eta = integrate(y * p_eta, (y, y_interval.start, y_interval.end))
Expect_xi_eta = Matrix([Expect_xi, Expect_eta])
    \end{sympycode}
    \[
        \begin{aligned}
            \Expect \xi
             & = \int\limits_{-\infty}^{+\infty} x \cdot p_\xi \, dx
            = \int\limits_{\sympy{x_interval.start}}^{\sympy{x_interval.end}} x \cdot \left(\sympy{p_xi} \right)\, dx
            = \sympy{Expect_xi},                                      \\
            \Expect \eta
             & = \int\limits_{-\infty}^{+\infty} y \cdot p_\eta \, dy
            = \int\limits_{\sympy{y_interval.start}}^{\sympy{y_interval.end}} y \cdot \left(\sympy{p_eta}\right) \, dy
            = \sympy{Expect_eta}.
        \end{aligned}
    \]

    \[
        \Expect(\xi,\eta) = \begin{pmatrix}
            \sympy{Expect_xi} \\
            \sympy{Expect_eta}
        \end{pmatrix}
    \]
    Найдем ковариационную матрицу:
    \begin{sympycode}
Expect_xi_square = integrate(x ** 2 * p_xi, (x, x_interval.start, x_interval.end))
Expect_eta_square = integrate(y ** 2 * p_eta, (y, y_interval.start, y_interval.end))
Variance_xi = Expect_xi_square - Expect_xi ** 2
Variance_eta = Expect_eta_square - Expect_eta ** 2
Expect_xi_eta = integrate(x * y * p_xi_eta, (y, y_interval_by_x.start, y_interval_by_x.end), (x, x_interval.start, x_interval.end))
Covariance_xi_eta = Expect_xi_eta - Expect_xi * Expect_eta
Covariance_matrix = Matrix([[Variance_xi, Covariance_xi_eta], [Covariance_xi_eta, Variance_eta]])
    \end{sympycode}
    Вычислим дисперсии $\xi$ и $\eta$:
    \[
        \begin{aligned}
            \Variance \xi
             & = \Expect \xi^2 - (\Expect \xi)^2
            = \int\limits_{-\infty}^{+\infty} x^2 \cdot p_\xi \, dx - (\Expect \xi)^2
            = \int\limits_{\sympy{x_interval.start}}^{\sympy{x_interval.end}} x^2 \cdot \left(\sympy{p_xi}\right) \, dx - \sympy{Expect_xi}^2
            = \sympy{Variance_xi},                                      \\
            \Variance \eta
             & = \int\limits_{-\infty}^{+\infty} y^2 \cdot p_\eta \, dy
            = \int\limits_{\sympy{y_interval.start}}^{\sympy{y_interval.end}} y^2 \cdot \left(\sympy{p_eta}\right) \, dy
            = \sympy{Variance_eta}.
        \end{aligned}
    \]
    Найдём ковариацию $\xi$ и $\eta$:
    \[
        \begin{aligned}
            \cov(\xi,\eta)
             & = \Expect(\xi \cdot \eta) - \Expect \xi \cdot \Expect \eta
            = \int\limits_{-\infty}^{+\infty} \int\limits_{-\infty}^{+\infty} x \cdot y \cdot p_{\xi,\eta} (x, y) \, dy \, dx
            \\
             & = \int\limits_{\sympy{x_interval.start}}^{\sympy{x_interval.end}} \, dx \int\limits_{\sympy{y_interval_by_x.start}}^{\sympy{y_interval_by_x.end}} x \cdot y \cdot \left(\sympy{p_xi_eta}\right)  \, dy
            = \sympy{Covariance_xi_eta}.
        \end{aligned}
    \]

    Матрица ковариаций:
    \[
        \Sigma = \sympy{Covariance_matrix}
    \]
    Найдем корреляционную матрицу. Для начала вычислим коэффициент корреляции:
    \begin{sympycode}
rho_xi_eta = Covariance_xi_eta / sqrt(Variance_xi * Variance_eta)
    \end{sympycode}
    \[
        \rho(\xi,\eta) = \frac{\cov(\xi,\eta)}{\sqrt{\Variance \xi \cdot \Variance \eta}}
        = \frac{\sympy{Covariance_xi_eta}}{\sqrt{\sympy{Variance_xi} \cdot \sympy{Variance_eta}} }
        = \sympy{rho_xi_eta}
    \]
    Корреляционная матрица:
    \[
        R = \begin{pmatrix}
            1                  & \sympy{rho_xi_eta} \\
            \sympy{rho_xi_eta} & 1
        \end{pmatrix}
    \]
    Найдем условное распределение $\xi$ при условии $\eta$:
    \[
        p_{\xi|\eta=y_0}(x) = \frac{p_{\xi, \eta}(x, y_0)}{p_\eta (y_0)}, p_\eta > 0
    \]
    \begin{sympycode}
x0, y0 = symbols("x_0 y_0")
p_xi1 = Piecewise((p_xi,  x_interval.contains(x)), (0, True))
p_eta1 = Piecewise((p_eta, y_interval.contains(y)), (0, True))
D1 = And(D, y<=5, x>=1)
p_xi_eta1 = Piecewise((p_xi_eta, D1), (0, True))
p_xi_cond_y0 = (p_xi_eta / p_eta).subs(y, y0).simplify()
p_xi_cond_eta = p_xi_cond_y0.subs(y0, eta).doit()
    \end{sympycode}
    \[
        \begin{aligned}
            p_{\xi|\eta=y_0}(x)
             & = \frac{p_{\xi, \eta}(x, y_0)}{{p_\eta (y_0)}}
            \\
            p_{\xi, \eta}(x, y_0)
             & = \begin{cases}
                     \sympy{p_xi_eta}, & \text{for } (x, y_0) \in D, \\
                     0,                & \text{otherwise}.
                 \end{cases}
            \\
            p_\eta(y_0)
             & = \sympy{p_eta1.subs(y, y0)}
            \\
            p_{\xi|\eta=y_0}(x)
             & = \begin{cases}
                     \sympy{p_xi_cond_y0}, & \text{for } x \in \sympy{x_interval_by_y.subs(y, y0)}, y_0 \in \sympy{y_interval}, \\
                     0,                    & \text{otherwise}.
                 \end{cases}
            \\
            p_{\xi|\eta} (x)
             & = \begin{cases}
                     \sympy{p_xi_cond_eta} & \text{for} x \in \sympy{x_interval_by_y.subs(y, eta)}, \\
                     0                     & \text{otherwise}.
                 \end{cases}
        \end{aligned}
    \]
    Найдем мат. ожидание и дисперсию случайной величины $\xi$ при условии $\eta$:
    \begin{sympycode}
x_interval_cond_eta = x_interval_by_y.subs(y, eta)
Expect_xi_cond_eta = integrate(x * p_xi_cond_eta, (x, x_interval_cond_eta.start, x_interval_cond_eta.end)).simplify()
Expect_xi_cond_eta_square = integrate(x ** 2 * p_xi_cond_eta, (x, x_interval_cond_eta.start, x_interval_cond_eta.end))
Variance_xi_cond_eta = Expect_xi_cond_eta_square - Expect_xi_cond_eta ** 2
Variance_xi_cond_eta = Variance_xi_cond_eta.simplify()
    \end{sympycode}
    \[
        \begin{aligned}
            \Expect(\xi|\eta)
             & = \int\limits_{-\infty}^{+\infty} x \cdot p_{\xi|\eta} (x) \, dx
            = \int\limits_{\sympy{x_interval_cond_eta.start}}^{\sympy{x_interval_cond_eta.end}} x \cdot \left(\sympy{p_xi_cond_eta}\right) \, dx
            = \sympy{Expect_xi_cond_eta}
            \\
            \Variance(\xi|\eta)
             & = \Expect(\xi^2|\eta) - (\Expect(\xi|\eta))^2
            = \int\limits_{-\infty}^{+\infty} x^2 \cdot p_{\xi|\eta} (x) \, dx - (\Expect(\xi|\eta))^2
            \\
             & = \int\limits_{\sympy{x_interval_cond_eta.start}}^{\sympy{x_interval_cond_eta.end}} x^2 \cdot \left(\sympy{p_xi_cond_eta}\right) \, dx - \left(\sympy{Expect_xi_cond_eta}\right)^2
            = \sympy{Variance_xi_cond_eta}
        \end{aligned}
    \]
\end{proof}

\newpage

\begin{problem}
Найти распределение $\mu $. Вычислить $\Expect \mu$ и $\Variance\mu$.
Построить график функции распределения $F_\mu(m).$
\end{problem}

\begin{proof}
    $\mu= \sympy{mu}$

    \begin{sympycode}
m = symbols("m")
mu_eq = mu.subs("eta", y).subs("xi", x) - m
\end{sympycode}

    Изобразим область D и прямую  $\sympy{mu_eq}$:
    \begin{figure}[h!]
        \centering
        \begin{tikzpicture}
            \begin{axis}[
                    xlabel=$x$,
                    ylabel=$y$,
                    xmin=-1.5, xmax=4.5,
                    ymin=-1.5, ymax=8.5,
                    axis lines=middle,
                    axis line style={->},
                    % ticks=none,
                    clip=true,
                    xtick={-1,0,1,2,3,4},
                    ytick={-1,0,1,2,3,4,5,6,7,8},
                    axis equal,
                    grid=both,
                ]
                \addplot[fill=blue!10] coordinates {(1,1) (3, 5) (3, 1) (1,1)};
                \addplot[domain=-1.5:4.5, red, dashed] {2*x - 3};
                \node[blue, right, rotate=63.435] at (axis cs: 1, 1.7) {$2x-1y-1=0$};
                \node[red, right, rotate=63.435] at (axis cs: 3, 1.7) {$-8x + 4y - m = 0$};
            \end{axis}
        \end{tikzpicture}
        \caption{Область $D$ и прямая $-8x + 4y - m = 0$}
        \label{fig:D_m}
    \end{figure}

\begin{sympycode}
# mu_eq = mu_eq.subs("m",4)
y_m = solve(mu_eq, y)[0]
x_m = solve(mu_eq, x)[0]
x1_m = x_m.subs(y, y1)
y2_m = y_m.subs(x, x2)
x_interval_m = Interval(x1_m, x2)
y_interval_by_x_m = Interval(y1, y_m)
m1 = solve(mu_eq, m)[0].subs(y, y_interval_by_x.start).subs(x, x_interval.end)
m2 = solve(mu_eq, m)[0].subs(y, y_interval_by_x.end).subs(x, x_interval.end)
m_interval = Interval(m1, m2)
F_mu = integrate(p_xi_eta,  (y, y_interval_by_x_m.start, y_interval_by_x_m.end), 
                            (x, x_interval_m.start, x_interval_m.end)).simplify()
    \end{sympycode}
        
    По определению функция распределения случайной величины $\mu$:


    \[ \begin{aligned}
        F_\mu(m)
         & = \Prob(\mu \leq m) = \Prob(\sympy{mu} \leq m)                                                                                  \\
         & = \int\limits_{-\infty}^{+\infty} \int\limits_{-\infty}^{+\infty} p_{\xi, \eta}(x, y) \cdot \ind{\sympy{mu} \leq m} \, dx \, dy
        = \int\limits_{\sympy{x_interval_m.start}}^{\sympy{x_interval_m.end}} \, dx
        \int\limits_{\sympy{y_interval_by_x_m.start}}^{\sympy{y_interval_by_x_m.end}} \sympy{p_xi_eta} \, dy
         & = \sympy{F_mu}.
    \end{aligned}
\]

\begin{sympycode}
F_mu1 = Piecewise((0, m <= m_interval.start),
                    (F_mu, m <= m_interval.end),
                    (1, True))
\end{sympycode}

В точке  $(x;y) = (3; 1)$ прямая $-8x + 4y - m = 0$ пересекает область $D$ при единственном значении $m = \sympy{m_interval.start}$.



    \[
        F_\mu(m) \sympy{F_mu1}
    \]



    \begin{figure}[h!]
        \centering
        \begin{tikzpicture}
            \begin{axis}[
                    xlabel=$m$,
                    ylabel=$F_\mu(m)$,
                    xmin=-25, xmax=2.5,
                    ymin=-0.5, ymax=1.5,
                    axis lines=middle,
                    axis line style={->},
                    % ticks=none,
                    clip=true,
                    % xtick={-4,-3,-2,-1,0,1,2},
                    ytick={0,1},
                    grid=both,
                ]
                \addplot[domain=-25:-20, blue, solid] {0};
                \addplot[domain=-20:-4, blue, solid] {x^2/256 + 5*x/32 + 25/16};
                \addplot[domain=-4:6.5, blue, solid] {1};
            \end{axis}
        \end{tikzpicture}
        \caption{График функции распределения $F_\mu(m)$}
        \label{fig:F_mu}
    \end{figure}

    \begin{sympycode}
p_mu = diff(F_mu, m)
Expect_mu = integrate(m * p_mu, (m, m_interval.start, m_interval.end))
Expect_mu = Expect_mu.simplify()
p_mu1 = Piecewise((p_mu, And(m <= m_interval.end, m>=m_interval.end)), (0, True))
Expect_mu_quare = integrate(m ** 2 * p_mu, (m, m_interval.start, m_interval.end))
Variance_mu = Expect_mu_quare - Expect_mu ** 2
\end{sympycode}
    Найдём плотность распределения случайной величины $\mu$:
    \[
        p_\mu(m) = \frac{d F_\mu(m)}{d m} = \sympy{p_mu1}
    \]

    Найдём математическое ожидание и дисперсию случайной величины $\mu$:
    \[
        \begin{aligned}
            \Expect \mu
             & = \int\limits_{-\infty}^{+\infty} m \cdot p_\mu(m) \, dm
            = \int\limits_{\sympy{m_interval.start}}^{\sympy{m_interval.end}} m \cdot \left(\sympy{p_mu}\right) \, dm
            = \sympy{Expect_mu}
            \\
            \Variance \mu
             & = \Expect \mu^2 - (\Expect \mu)^2
            = \int\limits_{-\infty}^{+\infty} m^2 \cdot p_\mu(m) \, dm - (\Expect \mu)^2
            = \int\limits_{\sympy{m_interval.start}}^{\sympy{m_interval.end}} m^2 \cdot \left(\sympy{p_mu}\right) \, dm
            -  \left( \sympy{Expect_mu} \right)^2
            = \sympy{Expect_mu_quare} - \left( \sympy{Expect_mu} \right)^2
            = \sympy{Variance_mu}
        \end{aligned}
    \]
\end{proof}	% для удобства создаём по аналогии файлы ihw1.tex, ihw2.tex, etc
% и просто меняем имя при компиляции
\end{document}