\problemset{Теория вероятностей и математическая статистика}
\problemset{Индивидуальное домашнее задание №2}	% поменяйте номер ИДЗ

\renewcommand*{\proofname}{Решение}

%%%%%%%%%%%%%% ЗАДАНИЕ №1 %%%%%%%%%%%%%%
%% Условие задания №1
\begin{problem}
Распределение случайной величины $\xi$ задано таблицей:

\[
  \begin{tabular}{ccc}
    k                & 2   & 3   \\
    $\textup{p}_{k}$ & 1/2 & 1/2 \\
  \end{tabular}
\]

Вычислить $\Expect\xi$, $\Variance\xi$, $\Entropy\xi$ (в натах). Вычислить распределение $\eta = |\xi-1| ^ {3/2}.$ Построить графики функций распределений $F_\xi(x)$ и $F_\eta(y).$
\end{problem}

%% Решение задания №1
\begin{proof}

  \begin{enumerate}
    \item Найдем математическое ожидание случайной величины $ \xi $:
          \[
            \Expect\xi = \sum_{k} k \cdot \textup{p}_{k} = 2 \cdot \frac{1}{2} + 3 \cdot \frac{1}{2} = \frac{5}{2}.
          \]

    \item Найдем дисперсию случайной величины $ \xi $:
          \[
            \Variance\xi = \Expect\xi^{2} - (\Expect\xi)^{2} = \sum_{k} k^{2} \cdot \textup{p}_{k} - \left( \frac{5}{2} \right)^{2} = 2^{2} \cdot \frac{1}{2} + 3^{2} \cdot \frac{1}{2} - \left( \frac{5}{2} \right)^{2} = \frac{1}{4}.
          \]

    \item Найдем энтропию случайной величины $ \xi $:
          \[
            \Entropy\xi = - \sum_{k} \textup{p}_{k} \cdot \ln \textup{p}_{k} = - \frac{1}{2}  \ln \frac{1}{2} - \frac{1}{2}  \ln \frac{1}{2} = \ln 2.
          \]

    \item Найдем распределение случайной величины $ \eta = |\xi - 1|^{3/2} $:

          Носитель случайной величины $ \supp \xi = \{2, 3\} $.

          Носитель случайной величины $ \supp \eta = \{1, 2\sqrt{2}\} $:

          \subitem При $ k = 2 $: $ \eta = |2 - 1|^{3/2} = 1^{3/2} = 1 $.

          \subitem При $ k = 3 $: $ \eta = |3 - 1|^{3/2} = 2^{3/2} = 2\sqrt{2} $.

          Таким образом, распределение случайной величины $ \eta $ задается таблицей:

          \[
            \begin{tabular}{ccc}
              $y$              & $1$           & $2\sqrt{2}$    \\
              $\textup{p}_{y}$ & $\frac{1}{2}$ & $\frac{1}{2} $ \\
            \end{tabular}
          \]

    \item Построим графики функций распределений $ F_{\xi}(x) $ и $ F_{\eta}(y) $:

          Функция распределения $ F_{\xi}(x) $ задается следующим образом:

          \[
            F_{\xi}(x) =
            \begin{cases}
              0,   & x \in (-\infty, 2], \\
              1/2, & x \in (2, 3],       \\
              1,   & x \in (2, +\infty).
            \end{cases}
          \]

          Функция распределения $ F_{\eta}(y) $ задается следующим образом:

          \[
            F_{\eta}(y) =
            \begin{cases}
              0,   & y \in (-\infty, 1],         \\
              1/2, & y \in (1, 2\sqrt{2}],       \\
              1,   & y \in (2\sqrt{2}, +\infty).
            \end{cases}
          \]

          Графики функций распределений $ F_{\xi}(x) $ и $ F_{\eta}(y) $ изображены на рисунках~\ref{fig:hw2:ind:1} и~\ref{fig:hw2:ind:2}.

          \begin{figure}[h!]
            \centering
            \begin{tikzpicture}
              \begin{axis}[
                  xlabel=$x$,
                  ylabel=$F_{\xi}(x)$,
                  xmin=-2, xmax=4,
                  ymin=0, ymax=1.5,
                  axis lines=middle,
                  clip=false,
                  extra x ticks={0},
                  xtick={-1, 0, 1, 2, 3},
                  ytick={0, 0.5, 1},
                  yticklabels={0, $\frac{1}{2}$, 1},
                  width=8cm,
                  height=6cm
                ]
                \addplot[blue, domain=-2:2] {0};
                \addplot[blue, domain=2:3] {0.5};
                \addplot[blue, domain=3:4] {1};
                \addplot[blue, mark=*, mark options={fill=white}, only marks] coordinates {(3,1)};
                \addplot[blue, mark=*, mark options={fill=white}, only marks] coordinates {(2,0.5)};
              \end{axis}
            \end{tikzpicture}
            \caption{График $F_{\xi}(x)$}
            \label{fig:hw2:ind:1}
          \end{figure}

          \begin{figure}[h!]
            \centering
            \begin{tikzpicture}
              \begin{axis}[
                  xlabel=$y$,
                  ylabel=$F_{\eta}(y)$,
                  xmin=-2, xmax=4,
                  ymin=0, ymax=1.5,
                  axis lines=middle,
                  clip=false,
                  extra x ticks={0},
                  xtick={-1, 0, 1, 2, 2.82, 3},
                  ytick={0, 0.5, 1},
                  xticklabels={-1, 0, 1, 2, , 3},
                  yticklabels={0, $\frac{1}{2}$, 1},
                  width=8cm,
                  height=6cm
                ]
                \addplot[blue, domain=-2:1] {0};
                \addplot[blue, domain=1:2.82] {0.5};
                \addplot[blue, domain=2.82:4] {1};
                \addplot[blue, mark=*, mark options={fill=white}, only marks] coordinates {(1,0.5)};
                \addplot[blue, mark=*, mark options={fill=white}, only marks] coordinates {(2.82,1)};
              \end{axis}
            \end{tikzpicture}
            \caption{График $F_{\eta}(y)$}
            \label{fig:hw2:ind:2}
          \end{figure}

  \end{enumerate}
\end{proof}

%%%%%%%%%%%%%% ЗАДАНИЕ №2 %%%%%%%%%%%%%%
%% Условие задания №2
\begin{problem}
Дана функция распределения абсолютно непрерывной случайной величины $\xi{:}$

\[
  F_\xi(x)=\left\{\begin{matrix}0, x\leqslant0,\\\sin(2x), x\in(0,C],\\1, x>C.\end{matrix}\right.
\]
Вычислить $C,\quad\mathbb{E}\xi,\quad\mathbb{D}\xi,\quad\mathbb{H}\xi\quad($в натах). Вычислить распределение $\eta=\sin(3\xi).$ Построить графики функций распределений $F_\xi(x)$ и $F_\eta(y).$
\end{problem}

%% Решение задания №2
\begin{proof}
  \begin{enumerate}
    \item Найдем константу $ C $:
          \[
            F_{\xi}(C) = 1 \Rightarrow \sin(2C) = 1 \Rightarrow 2C = \frac{\pi}{2} \Rightarrow C = \frac{\pi}{4}.
          \]

    \item Плотность распределения случайной величины $ \xi $:
          \[
            p_{\xi}(x) = \frac{dF_{\xi}(x)}{dx} = 2 \cos(2x) \cdot \ind{0 < x \leqslant \frac{\pi}{4}}.
          \]

    \item Найдем математическое ожидание случайной величины $ \xi $:
          \[
            \Expect\xi = \int_{0}^{\frac{\pi}{4}} x \cdot 2 \cos(2x) \, dx = \left. x \cdot \sin(2x) \right|_{0}^{\frac{\pi}{4}} - \int_{0}^{\frac{\pi}{4}} \sin(2x) \, dx = \frac{\pi}{4} \cdot \sin\left( \frac{\pi}{2} \right) - \left. \frac{1}{2} \cos(2x) \right|_{0}^{\frac{\pi}{4}} = \frac{\pi}{4} - \frac{1}{2}.
          \]

    \item Найдем дисперсию случайной величины $ \xi $:
          \[
            \Variance\xi = \Expect\xi^{2} - (\Expect\xi)^{2} = \int_{0}^{\frac{\pi}{4}} x^{2} \cdot 2 \cos(2x) \, dx - \left( \frac{\pi}{4} - \frac{1}{2} \right)^{2} = \frac{\pi^{2}}{16} - \frac{\pi}{2} - \frac{\pi^2}{16} - \frac{-\pi+1}{4} = \frac{3}{4} - \frac{\pi}{4}.
          \]

    \item Найдем энтропию случайной величины $ \xi $:
          \[
            \Entropy\xi = - \int_{0}^{\frac{\pi}{4}} \sin(2x) \cdot \ln \sin(2x) \, dx = \frac{1}{2} - \frac{1}{2}\ln 2.
          \]

    \item Найдем распределение случайной величины $ \eta = \sin(3\xi) $:

          Носитель случайной величины $ \supp \xi = [0, \frac{\pi}{4}] $.

          Носитель случайной величины $ \supp \eta = \{0\} \cup \{ \sin(3x) \mid x \in (0, \frac{\pi}{4}] \} = [0, 1]$.

          \begin{figure}[h!]
            \centering
            \begin{tikzpicture}
              \begin{axis}[
                  xlabel=$x$,
                  ylabel=$y$,
                  xmin=0, xmax=pi/3,
                  ymin=0, ymax=1.5,
                  axis lines=middle,
                  clip=false,
                  extra x ticks={0},
                  xtick={0, pi/12, pi/6, pi/4},
                  xticklabels={0, $\pi/12$, $\pi/6$, $\pi/4$},
                  ytick={0, 0.5, 1},
                  yticklabels={0, $\frac{1}{2}$, 1},
                  width=8cm,
                  height=6cm
                ]
                \addplot[cyan, domain=0:pi/6] {sin(3*deg(x))};
                \addplot[magenta, domain=pi/6:pi/4] {sin(3*deg(x))};
              \end{axis}
            \end{tikzpicture}
            \caption{$\sin 3x$}
            \label{fig:hw2:ind:5}
          \end{figure}

          Разобьем носитель случайной величины $ \eta $ на два интервала, где функция $\sin 3x$ монотонна: $ [0, \frac{\pi}{6}] $ и $ (\frac{\pi}{6}, \frac{\pi}{4}] $.

          \begin{enumerate}
            \item При $ x \in [0, \frac{\pi}{6}] $, $ y \in [0, 1]$:
                  \[
                    \begin{cases}
                      g_1(x) = \sin 3x, \\
                      g_1^{-1}(y) = \frac{1}{3}\arcsin y.
                    \end{cases}
                  \]
            \item При $ x \in (\frac{\pi}{6}, \frac{\pi}{4}] $, $ y \in [\frac{\sqrt 2}{2}, 1]$:
                  \[
                    \begin{cases}
                      g_1(x) = \sin 3x, \\
                      g_1^{-1}(y) = \frac{1}{3}\arcsin y + \frac{\pi}{3}.
                    \end{cases}
                  \]
          \end{enumerate}

          

    \item Построим графики функций распределений $ F_{\xi}(x) $ и $ F_{\eta}(y) $:
    \item Функция распределения $ F_{\xi}(x) $ задается следующим образом:
          \[
            F_{\xi}(x) =
            \begin{cases}
              0,        & x \leqslant 0,            \\
              \sin(2x), & x \in (0, \frac{\pi}{4}], \\
              1,        & x > \frac{\pi}{4}.
            \end{cases}
          \]

    \item Функция распределения $ F_{\eta}(y) $ задается следующим образом:
    \item Графики функций распределений $ F_{\xi}(x) $ и $ F_{\eta}(y) $ изображены на рисунках~\ref{fig:hw2:ind:3} и~\ref{fig:hw2:ind:4}.

          \begin{figure}[h!]
            \centering
            \begin{tikzpicture}
              \begin{axis}[
                  xlabel=$x$,
                  ylabel=$F_{\xi}(x)$,
                  xmin=-2, xmax=4,
                  ymin=0, ymax=1.5,
                  axis lines=middle,
                  clip=false,
                  extra x ticks={0},
                  xtick={-1, 0, 1, 2, 3},
                  ytick={0, 0.5, 1},
                  yticklabels={0, $\frac{1}{2}$, 1},
                  width=8cm,
                  height=6cm
                ]
                \addplot[blue, domain=-2:0] {0};
                \addplot[blue, domain=0:pi/4] {sin(2*deg(x))};
                \addplot[blue, domain=pi/4:4] {1};
              \end{axis}
            \end{tikzpicture}
            \caption{График $F_{\xi}(x)$}
            \label{fig:hw2:ind:3}
          \end{figure}
  \end{enumerate}
\end{proof}
